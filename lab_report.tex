\documentclass[12pt]{article}

%%%%%%%%%%%%%%%%%%%%%%% Don't change anything in here. This space is called the preamble, it is where you tell the computer to load the proper LaTeX packages to perform the math and formatting desired. 

\usepackage{physics} 
\usepackage{siunitx} 
\usepackage{enumerate} 
\usepackage{pgfplots}
\usepackage{pgfplotstable}
\usepackage{tikz,pgfplots}
\usepackage{amsmath}  %I added this so that you can use the align tool for equations!
\usepackage{wasysym} %This package allows you to put emojis in your paper!!!!
	%wasysym: \smiley{} \frownie{} see http://milde.users.sourceforge.net/LUCR/Math/mathpackages/wasysym-symbols.pdf for list of most symbols available in this package
	
\usepackage{geometry}
 \geometry{
 a4paper,
 total={170mm,257mm},
 left=20mm,
 top=20mm,
 }

\pgfplotsset{compat=1.14}
%%%%%%%%%%%%%%%%%%%%%%%%% Again, Don't change anything Above %%%%%%%%%%%%%%%%%%%%

\begin{document}

\title{PHSA0030  Mini Project - Brazilian nuts simulation using a Monte Carlo approach} %Title should be concise and to the point  
\author{Joel Pérez Ferrer} %Your name first

\date{\today}  % This will automatically put today's date in the report

\maketitle  %this command makes the title

%%%% to use this template, please copy-paste the entire thing into a new document and save it so you have it!

%%%%% If you want to omit something in this lab, place a % sign to the left of it and it won't show up on the lab, like this line!

\begin{abstract} 
  The formal report should contain: an abstract; the background to the project; a summary of the work done (though not in a narrative form); the results and a discussion of their significance; any conclusions that you have drawn, and possibly suggestions for future work, or where you would take the project given more time; and a bibliography.


\end{abstract}

\section{Background} % to the project

Empirical observations show that if a container of a mix of nuts of different sizes is shaken thoroughly, the larger pieces (such as the brazil nut kernels) tend to rise to the top. In this project we will be using a Monte Carlo method is an effort to simulate this physical situation. More specifically, we will be using the Metropolis approach as described in the course from which this assignment is part of.

\subsection{Monte Carlo simulation in a nutshell}
In our simulation we will start with a distribution of elements, select a random element and displace it a random distance in a random direction. If the movement overall lowers the energy of the system we will accept it as valid. But in order to avoid local minima we will introduce a probability for an energetically \textit{worse} movement to be done. The probability used is $ p = e^{\frac{\Delta E}{k_B T}} $ Where the energy change is $\Delta E$, thus the probability decreases as the energy change increases. The probability also increases as the temperature increases, which can be observed in nature.


\section{Summary} % of the work done, though not in a narrative form
In the simulation we assumed that the brazilian nuts were two dimensional disks and the regular nuts as disks of half of the radius of the disks representing the brazilian nuts. The box has a limited horizontal width but we assume it to be infinitely tall.

\subsection{Comment on the tools used}
For this project, Python 3.7 was used, along with the libraries numpy for the data structures and calculations and matplotlib for plotting and graphics. Additionally, Jupyter notebooks were used as logbooks detailing the progress and stages of the project.

\subsection{Initial configuration of the system}
As per the guidelines of the assignment, the initial configuration of the system was a brazilian nut at the bottom and then regular nuts spread randomly in the x coordinate in increasingly high positions. 

\subsection{Monte Carlo method}
The Montecarlo approached was implemented as follows: First a particle in the system is chosen at random. Then a step size is calculated at random based on the value for the largest nearest neighbor of the system. The angle of the displacement is also chosen randomly. If the movement is energetically better (i.e. $\Delta y < 0$) it is accepted, otherwise it is checked against a probability $ p = e^{\frac{\Delta E}{k_B T}} $. Then the move is tested for collisions with the walls or with other particles. If the move fulfilled all these tests it is be taken as \textit{accepted} and the system is updated. The change in energy achieved by this move is accumulated in some $\Delta E$ variable that keeps track of the change in energy achieved every 1000 attempts. As the particles become incrementally closer to each other and to the bottom of the bottom of the box, both the value for the largest nearest neighbor drops and the frequency of accepted moves drop substantially. This results eventually results in $\Delta E$ not reaching an arbitrary threshold and thus th system being considered \textit{settled}. From there it can be tested whether the big nuts are on top and otherwise the system may be \textit{shaken}(i.e. all the $y$ values multiplied by $2$) and the process started all over. 

\section{Results}

\subsection{Relation between $n_{nut}$ and $n_{shakes}$}
Evidently, the increase of the number of nuts in the system $n_{nut}$ resulted in an overall increase in the number of shakes $n_{shakes}$ needed in order to take the brazilian nut to the top.

However, a factor that was highly relevant in order to judge the value of the investigations presented in the following subsections was the standard deviation $\sigma_{shakes}$ between consecutive simulations of the same configurations. This value was revealed to decrease as $n_{nut}$ increases. Thus a high value for $n_{nut}$ is preferred in order to minimize the uncertainity in our results. However, the computational times needed to simulate systems with very large $n_{nut}$ values quickly start to become non-trivial. 

\subsection{Effects of $r_{brazilian}/r$}

\subsection{Study of the energy}

\subsection{Temperature and $p$}

\section{Future work} % where you the move take the project given more time

\begin{description}
  \item[$\bullet$] 61cm ramp
  \item[$\bullet$] stack of textbooks
  \item[$\bullet$] ball
  \item[$\bullet$] meter stick
  \item[$\bullet$] tape
  \item[$\bullet$] timer
  \item[$\bullet$] Logger Pro
  \item[$\bullet$] motion sensor
\end{description}
%%%%%%%%%%%%%%%%%%%%%%%%%%%%%%%%%%%%%%%%%%%%%%



%%%%%%%%%%%%%%%%%%%%%%%%%%%%%%%%%%%%%%%%%%%%%%%%%%


%%%%%%%%%%    Using Graphs/pictures in LaTeX    %%%%%%%%%%%%%%%%%%%%%%%%%%
%In this section you will find the code for inserting a graph into LaTeX. My suggestion is to use excel to create your graph, so that you can include trend lines, error bars (set to 2 sigma standard deviation or to a set amount, e.g. 0.2s to include roughly for human error effects due to timing. You can also use excel to plot experimental data (with error bars) and theoretical data, which allows you to easily see if the theoretical data falls within the region of acceptable error. 

%Once your graph is created, take a screen shot using print screen (prt scn on most computers). Paste into paint and crop out your graph and save as a jpg or png file. Title your picture as something simple to reference later, for example "graph1"

%You will then click the "project" tab in the upper left-hand corner of overleaf, click the drop-down arrow "main.tex" and click upload to upload your picture. From there, use the code below to insert the picture into your report. 

%technique #1: Basic way
%\includegraphics[width=\textwidth]{YOUR-FILE-NAME-HERE}

%technique #2: slightly more advanced way which allows formatting, sizing, cen

%\begin{figure}[h!]
 % \caption{A picture of a gull.}
  %\centering
  %\includegraphics[width=0.5\textwidth]{Image File Name}
%\end{figure}

%%%%%%%%%%%%%%%%%%%%%%%%%%%%%%%%%%%%%%%%%%%%%%%%%%%%%%%%%%%%%%%%%%%%%%%%%%%%%%%%





%%%%%%%%%%%%%%%%%%%%%%%%%%%%%%%%%%%%%%%%%%%%%%%%%%%%%%%%%%%%%%%%%%%%%%%%%%%%%%%%%
%Here are some sample equations showing how to use the align tool to number and align math in your report

\begin{align} %Note that in align, you are in "math mode" and thus you don't need the $ between your math
K &= U_g \\
mgh &=\frac{1}{2}mv^2 \\
(10)(.225) &=\frac{1}{2}v^2 \nonumber \\  %by using \nonumber, you can take off the label on the side of an equation, which is nice so that steps of math with work aren't numbered. It is entirely your choice if you want to use this or not. 
4.5 &=v^2 \nonumber \\
2.12 \frac{m}{s} &=v 
\end{align}
%The align tool allows you to number your equations, which is super useful for referencing math in your work. I.e. you could say "see equation (1) for details." 

%Note: the \\ is to make a new line. Whatever you want aligned, just use & to the left, for example see the = above. Finally, don't use \\ on the last line or it will create a blank bottom equation
%%%%%%%%%%%%%%%%%%%%%%%%%%%%%%%%%%%%%%%%%%%%%%%%%%%%%%%%%%%%%%%%%%%%%%%%%%%%%%%%%%%



%~~~~may the f=ma be with you!~~~~~~~~Mr. C 



%%%%%%%%%%% Here is how to include a picture in your report   %%%%%%%%%%%%%%%%%%%%%
% \includegraphics[width=.3\textwidth]{Picture13}
% In this example, Picture2.png is the name of the picture file. For every picture that you want, you first have to save the picture to your computer, then give it a name like Picture2.png 
%%%%%%%%%%%%%%%%%%%%%%%%%%%%%%%%%%%%%%%%%%%%%%%%%%%%%%%%%%%%%%%%%%%%%%%%%%%%%%%%%%%%



\section{Analysis}
	This is the most important part of the lab report; it is where you
analyze the data. In this section you will interpret your results. You need to look at your data and decide if the hypothesis was supported or contradicted by your data.  \\ %Note: the \\ gives you a line of space
    
    Your discussion should include the following at a minimum. [1] What is the relationship between your measurements and your final results? [2] What trends were observable? [3] What can you conclude from the graphs that you made? [4] How did the independent variables affect the dependent variables? (For example, did an increase in a given measured (independent) variable result in an increase or decrease in the associated calculated (dependent) variable?) \\ 

Then describe how your experimental results substantiate/agree with the theory. (This is not a single statement that your results agree or disagree with theory.) When comparison values are available, discuss the agreement using either uncertainty and/or percent differences. This leads into the discussion of the sources of error. Your discussion should include the calculation of averages and standard deviation to be able to describe precision of experiment. All data points should be plotted $\pm$ two standard deviations and compared with theoretical data to interpret accuracy. It is ok to say that your results were inconclusive. It is important to cite all possible sources of error and state specifically how you believed they affected the collected data. If you get a result or an uncertainty that is ridiculous (or just really big/small), show that you have noticed and thought about it, not just copied a number from your calculator and moved on. \\

For example, when rolling a ball down a ramp, you may not have taken in account the effects of rolling friction or the fact that some gravitational potential energy is converted into rotational kinetic energy. Both of these the move cause the overall time for the ball to roll down the ramp to increase.


\section{Conclusion}
    The conclusion should connect to the introduction and re-state the relevance and importance of the experiment. Its a nice touch to sometimes make historical connections in this part of the report as well. It is always good to end on a note stating the importance of your findings, the connections to other topics in physics and science, and opportunities for future extensions/research/experiments in the subject. Remember to report your results with correct units and uncertainties, for example $g=9.7 \pm 0.2 m \cdotp s^{-2}$. 
    
%%%%%%%%%%%%%%%%%%%%%%%%%%%%%%%%%%%%%%%%%%%%%%%%%%%%%%%%%%%%%%%%%%%%%%%%%%%%%%%%%%%%%
%%%% This is the Bibliography where you will cite your sources used in the paper %%%%

\begin{thebibliography}{0}

	%Each item starts with a \bibitem{} command and the details thereafter.
	
	\bibitem{1} Cite your first source here
	\bibitem{2} Cite another source
	
    %%% The 1,2 etc. are used to cite in text. See up in the intro for an example
    %%% When you want to cite in your cite, type in \cite{} wherever you want
\end{thebibliography}
    
%%%%%%%%%%%%%%%%%%%%%%%%%%%%%%%%%%%%%%%%%%%%%%%%%%%%%%%%%%%%%%%%%%%%%%%%%%%%%%%%%%%%    


%%%%%%%%%%%%%%%%%%%%%%%%%%%%%%%%%%%%%%%%%%%%%%%%%%%%%%%%%%%%%%%%%%%%%%%%%%%%%%%%%%%%%

% Here are some equations and other useful things, feel free to copy and paste into your lab report :) %

% Note: It is super easy to look up equations/constants already formatted in LaTeX online. Here are a few websites I like to use:

% LaTeX Tutorial: http://pages.physics.cornell.edu/sps/pages/resources/latex.html
	%NOTE: This contains both pdf and text (code) of each document, which includes guides, lab 			report templates, and lots of other good stuff!

% LaTeX Cheat Sheet: http://wch.github.io/latexsheet/

% Equations: http://www.equationsheet.com/sheets/Equations-5.html

% Constants, symbols, letters, etc:  http://www.rpi.edu/dept/arc/training/latex/LaTeX_symbols.pdf

% AP Physics Calculus Reference Table: https://secure-media.collegeboard.org/digitalServices/pdf/ap/physics-c-tables-and-equations-list.pdf

% AP Physics Trig Reference Table: https://secure-media.collegeboard.org/digitalServices/pdf/ap/ap-physics-1-equations-table.pdf

% Regents Physics Reference Table: http://www.p12.nysed.gov/assessment/reftable/physics-rt/physics06tbl.pdf

% Here are three good sources to use other then this for considering how to write a good lab report. Much of this guide was gleaned from them
	
    %http://web.mit.edu/8.13/www/Samplepaper/simple-zipped/simple-paper.pdf 				-MIT Physics Lab (really, really good template!) 
    
    % http://pages.physics.cornell.edu/sps/pages/resources/LatexSession/Exercises/LabReport.pdf -Cornell Template
    % https://engineering.purdue.edu/ME588/LabManual/report_format.pdf 							-Purdue Engineering Template 
    % http://physics.columbia.edu/files/physics/content/1291_report_format_and_example.pdf 		-Columbia University Template
    % https://www.baylor.edu/physics/doc.php/110769.pdf											-Baylor University Template
    % www.nd.edu/~hgberry/Fall2012/Guidelines.docx												-Notre Dame Template
    %http://www.esf.edu/iq/colloquium/documents/LabReportnotes.pdf								-SUNY ESF Template
    %http://writing.engr.psu.edu/workbooks/laboratory.html										-Virginia Tech Template
    %https://projects.ncsu.edu/labwrite/index_labwrite.htm										-SUPER in-depth guide to writing lab reports
    
    %https://gist.github.com/dcernst/1827406													-Template for completing Math homework in LaTeX
    %https://joshldavis.com/2014/02/12/doing-your-homework-in-latex/							-More about math homework in LaTeX
    
    
    
%Purdue University Online Writing Lab-OWL: https://owl.english.purdue.edu/ Use this to generate citations!

%Basically, if you get stuck, just google "Latex ______" for whatever you need and look through the LaTeX stackexchange or wiki article to find and copy/paste what you need





%	Here are some common equations we use in class. I will continue to update as we continue throughout the year. I will attempt to organize by the order we learn the topics from oldest at the top to newest at the bottom. Go ahead and copy/paste as needed in your report

%%%%%%%%%%%%%%%%%%%%%%%%%% Basic Calculus %%%%%%%%%%%%%%%%%%%%%%%%%%%%%%%%%%%%%

%		   $$\frac{\mathrm{d}}{\mathrm{d}x}C=0$$
%           $$\frac{\mathrm{d}}{\mathrm{d}x}Cx=C$$
%           $$\frac{\mathrm{d}}{\mathrm{d}x}x=1$$           
%           $$\frac{\mathrm{d}}{\mathrm{d}x}x^n = nx^{n-1} $$  	-power rule
% 		   $$\frac{\mathrm{d}}{\mathrm{d}x}fg= fg'+f'g$$		-product rule
%           $$\frac{\mathrm{d}}{\mathrm{d}x}f(g(x))=f'(g(x))g'(x)$$	-chain rule
%           $$\frac{\mathrm{d}}{\mathrm{d}x} \sin{x} = \cos{x}$$	
%           $$\frac{\mathrm{d}}{\mathrm{d}x} \cos{x} = -\sin{x}$$
           
%           $$\int k \mathrm{d}x = kx+C$$		-integral of a constant
%           $$\int x^n \mathrm{d}x= \frac{1}{n+1}x^{n+1}+C$$	-power rule for integrals
%           $$\int \cos{u}\mathrm{d}u = \sin{u} + C$$	
%           $$\int \sin{u}\mathrm{d}u = -cos{u} + C$$	

% see https://reu.dimacs.rutgers.edu/Symbols.pdf for a nice list of math LaTeX symbols

%%%%%%%%%%%%%%%%%%%%%%%%%%%%  Kinematics %%%%%%%%%%%%%%%%%%%%%%%%%%%%%%%%%%%%%%%

%			$$\bar{v}=\frac{d}{t}$$ 								-average speed
%			$$v=\frac{\mathrm{d}x}{\mathrm{d}t}$$					-instantaneous velocity definition
%			$$a=\frac{\mathrm{d}v}{\mathrm{d}t}$$					-instantaneous acceleration definition
%			if acceleration is constant, then:
%				$${x_f}={x_i}+{v_i}t+\frac{1}{2}a{t^2}$$  			-free-fall equation
%				$${v_f}={v_i}+at$$									-find new velocity
%				$${{v_f}^2}={{v_i}^2}+2a({x_f}-{x_i})$$				-equation without time

%%%%%%%%%%%%%%%%%%%%%%%%%%%%% Newton's Laws %%%%%%%%%%%%%%%%%%%%%%%%%%%%%%%%%%%%%%

%			$$F_{net}=ma=m\frac{\mathrm{d}v}{\mathrm{d}t}=m\frac{\mathrm{d}^{2}x}{\mathrm{d}{t^2}}$$	-Newt's 2nd Law
%			$$F_f=\mu F_n$$											-Friction 
%			$$w=mg$$												-Weight

%%%%%%%%%%%%%%%%%%%%%%%%%%%%%%%% Work, Power, Energy %%%%%%%%%%%%%%%%%%%%%%%%%%%%%%%%%%%

%			$$W=\Delta E = \int F dx = Fd  \hspace{3pt} \text{(if F constant)} $$ 	-Work-Energy Theorem
%			$$Power=\frac{\mathrm{d}E}{\mathrm{d}t}=\frac{\mathrm{d}W}{\mathrm{d}t} = \frac{Fd}{t} = F \bar{v}$$		-Power
%			$$K=\frac{1}{2}mv^2$$													-Kinetic Energy
%			$$U_g=mgh$$																-Gravitational Potential Energy
%			$$U_e=\frac{1}{2}kx^2$$													-Spring Potential Energy
%			$$F_e=kx$$																-Hooke's Law
%			$$F=-\frac{\mathrm{d}U}{\mathrm{d}x}$$									-Force is derivative of Potential

%%%%%%%%%%%%%%%%%%%%%%%%%%%%%%%%%%%%% Momentum, Center of Mass %%%%%%%%%%%%%%%%%%%%%%%%%%%%%%%%%%%

% 			$$p=mv$$								-definition of momentum
%			$$F=\frac{\mathrm{d}p}{\mathrm{d}t}
%			$$ft=\Delta{p}$$ 						-trig version of impulse momentum theorem
%			$$J=\int F \mathrm{d}t = \Delta{p}$$	-calc version of impulse momentum theorem
%			$$p_{before}=p_{after}$$				-Conservation of Momentum
%			$$X_{c.o.m}=\frac{\Sigma x_i m_i}{M}$$  -x-coordinate of Center of Mass
%       	$$Y_{c.o.m}=\frac{\Sigma y_i m_i}{M}$$  -y-coordinate of Center of Mass


%%%%%%%%%%%%%%%%%%%%%%%%%%%%%%%%%%% Rotational Kinematics %%%%%%%%%%%%%%%%%%%%%%%%%%%%%%%%%%%%%%%

%		$$\omega=\frac{\mathrm{d}\theta}{\mathrm{d}t}$$ 			-definition of angular speed (rad/sec)
%		$$\alpha=\frac{\mathrm{d}\omega}{\mathrm{d}t}=\frac{\mathrm{d}^2\theta}{\mathrm{d}t^2}$$ 			-definition of angular acceleration
%		$$v=r\omega$$												-angular/linear velocity connection
%		$$S=r \theta$$												-arc length/angle connection
%		if angular acceleration is constant, then:
%			$$\omega_f=\omega_i+\alpha t$$   	-find angular velocity
%			$$\theta=\theta_i + \omega_i t + \frac{1}{2} \alpha t^2$$ 		-angular "free-fall" equation
%			$${\omega_f}^2 = {\omega_i}^2 + 2\alpha(\theta_f - \theta_i)$$ 	-angular equation w/o time

%%%%%%%%%%%%%%%%%%%%%%%%%%%%%%%% Rotational Dynamics + Gravitation %%%%%%%%%%%%%%%%%%%%%%%%%%%%%%%%%%%%%%%%%%%

%		$$\tau=r \times F = rF\sin{\theta}$$			-definition of torque
%		$$\tau = I \alpha$$								-Newt's 2nd Law for Rotation
%		$$a_c = v^{2}/r = {\omega^2}r$$					-centripetal acceleration
%		$$F_c=ma_c = m{\omega^2}r$$						-centripetal force
%		$$I=\int r^2 \mathrm{d}m = \Sigma m r^2$$		-Calculate Moment of Inertia of an Object
%		$$K_r=\frac{1}{2}I{\omega^2}$$					-Rotational Kinetic Energy
%		$$L= I\omega = r \times p = rp\sin{\theta}$$	-Angular Momentum
%		$$L_{before}=L_{after}$$						-Conservation of Angular Momentum

%		$$F_g = \frac{\text{G}m_1 m_2}{r^2}$$			-Newton's Law of Universal Gravitation
%		$$U_g= -\frac{\text{G}m_1 m_2}{r}$$				-Gravitational Potential Energy

%%%%%%%%%%%%%%%%%%%%%%%%% Vibrations, Simple Harmonic Motion, Sound %%%%%%%%%%%%%%%%%%%%%%%%%%%%%%	

%		$$v=f\lambda$$								-speed of a wave
%		$4T=\frac{2\pi}{\omega}=\frac{1}{f}$$		-period of a wave
%		$$x(t)=x_{max}\cos{(\omega t + \phi)}$$	    -wave equation
%		$$T_s=2\pi\sqrt{\frac{m}{k}}$$				-Period of an oscillating spring
%		$$T_p=2\pi\sqrt{\frac{l}{g}}$$				-Period of a Pendulum
%		$$\frac{\mathrm{d}^2 \smiley{}}{{dt}^2} - {\omega}^2 \smiley{} = 0$$ 	-General Simple Harmonic Motion Equation
%		
		
%%%%%%%%%%%%%%%%%%%%%%%%%%%%%%%%%%%%%% Fluid Mechanics %%%%%%%%%%%%%%%%%%%%%%%%%%%%%%%%%%%%%%%%%%%

%		$$\rho=m/V$$						-Density
%		$$P=F/A$$							-Definition of Pressure
%		$$P=P_i+\rho gh	$$					-Pressure change as a function of depth
%		$$\rho_1A_1v_1=\rho_2A_2v_2$$		-Continuity Equation for fluid flow
%		$$F_b=\rho gV_{displaced}$$			-Archimedes Principle
%		$$P_1+\frac{1}{2}\rho v_1^2 +\rho gh_1 = P_2+\frac{1}{2}\rho v_2^2 +\rho gh_2 $$ -Bernoulli Equation
%		

%%%%%%%%%%%%%%%%%%%%%%%%%%%%%%%%%%%%%% Thermodynamics %%%%%%%%%%%%%%%%%%%%%%%%%%%%%%%%%%%%%%%%%%%%%

%		$$\frac{\Delta Q}{\Delta t} = \frac{k A \Delta T}{l}$$ 		-Conduction Equation
%		$$Q = mc \Delta T$$											-Heat Flow Equation (sensible heat)
%		$$Q=mL_f$$													-Latent Heat of Fusion
%		$$Q=mL_v$$													-Latent Heat of Vaporization
%		$$ \Delta l = \alpha l_0 \Delta T$$							-Length expansion
%		$$ \Delta A = 2 \alpha A_0 \Delta T$$						-Area expansion
%		$$ \Delta V = 3 \alpha V_0 \Delta T$$						-Volume Expansion
%       $$PV=Nk_BT$$												-Ideal Gas Law
%		$$K=\frac{n}{2}k_BT$$										-Equipartition Theorem
			%if Ideal Gas:
            	% then $$K=\frac{3}{2}k_BT$$
%		$$W=-\int P \mathrm{d}V										-Thermodynamic Work on a gas (calc)
%		$$W=-P \DeltaV$$											-Thermodynamic Work on a gas Equation-trig
%		$$\Delta U= Q+W$$											-1st Law of Thermodynamics
%		$$\epsilon_{real} = 1- \frac{Q_c}{Q_h}$$					-Real Efficiency 
%       $$\epsilon_{theory} = 1- \frac{T_c}{T_h}$$					-Theoretical "Carnot" efficiency


%%%%%%%%%%%%%%%%%%%%%%%%%%%%%%%%%%%%%%% Misc. Useful Things %%%%%%%%%%%%%%%%%%%%%%%%%%%%%%%%%%%%%%%
% 
% \framebox{box}  This puts a box around something. Good for showing something important
% to quote someone, use the following template:
	%\begin{quotation}
	%``You miss 100\% of the shots you never take'' %note that quotes are formatted this way in LaTeX
	%-Michael Scott
	%\end{quotation}



% https://physics.info/equations/ is a good source for most common equations found in introductory physics (trig and calc based)




\end{document}
